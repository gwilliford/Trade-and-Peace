\subsection{Substantive Article 1: Economic Interdependence and the Management of Territorial Disputes}
The first proposed empirical chapter examines when states engage in peaceful conflict management attempts and successfully reach negotiated settlements over territorial claims. If my theory is correct, leaders should have more freedom to pursue accommodationist policies in disputes where the disputants have significant economic linkages with each other. As a result, leaders will be more likely to pursue peaceful settlements and more likely to agree to compromises. 
% Attempts
% Perception of weakness
Leaders who engage in peaceful conflict management attempts without the support of their domestic backers may be removed and replaced by individuals who will take a hard-line stance against the enemy.
% Not worth the time/resources/energy
In addition, engaging in negotiations with other states involves the investment of time, resources, and energy that may otherwise be directed at achieving other policy goals \citep{keohane2005}. As a result of these transaction costs, leaders do not have an incentive to engage in peaceful settlement attempts unless they believe that an agreed upon settlement can ultimately be approved and implemented by their domestic constituents. This produces the following hypothesis:

\begin{hypothesis} As the level of economic interdependence between two states increases, the probability that they engage in peaceful settlement attempts increases. \end{hypothesis}

% Settlement
In addition, domestic support for settlement influences whether two states can reach a mutually acceptable agreement. Although negotiations occur between states, the negotiators must be concerned about whether they can convince domestic constituencies to ratify and implement an agreement. This complicates the bargaining process, insofar as negotiators must find an agreement that is not only mutually acceptable to both states, but to both states' domestic audiences as well \citep{putnam1988}. In addition, if one state suspects that their opponent is likely to renege on an agreement due to the domestic costs associated with it, their opponent will find it difficult to commit credibly to upholding the agreement, which may preclude settlement \citep{fearon1998, putnam1988}.

\begin{hypothesis} As the level of economic interdependence between two states increases, domestic support for settlement increases, increasing the probability that they reach a negotiated settlement increases. \end{hypothesis}	
	% Concessions
		% \begin{hypothesis} As the level of economic interdependence between two states increases, the probability that states make significant territorial concessions increases. \end{hypothesis}
