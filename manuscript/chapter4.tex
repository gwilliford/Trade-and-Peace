


Chapter 1















The second empirical chapter concerns whether states ratify and implement the terms of an international agreement or abrogate their commitments. In the event that settlements are unpopular with domestic constituents, leaders may be unable to implement agreements. Three mechanisms may prevent states from adhering to agreements and thereby lead to failure.















First, even after leaders have signed agreements, domestic opposition to an agreed-upon settlement may prevent them from complying fully with the terms of an agreement. Although policymakers may genuinely intend to follow through on their commitments, they may be unable to do so if they do not have strong support among the winning coalition \citep{putnam1988}. Doing so may lead the winning coalition to remove the leader from power in order to renege on the agreement.















\citep{Introduction}















Domestic Support and Adherence















The preferences of domestic groups may also determine whether states ultimately comply with agreements once they have been signed. 















Leaders may be unable to convince domestic audiences that a particular agreement is beneficial and may therefore be unable to implement agreements after the fact.	















% Note: Consistent with this, \citet{simmons2002} found that states with higher levels of judicial independence are more likely to comply with settlements reached as the result of international arbitration or adjudication. This suggests that domestic populations in countries with strong rule of law may be more willing to accept the decisions of international legal bodies, even if they disagree with them.















Most directly, domestic actors will be unlikely to ratify agreements when they disagree on the consequences that















•	Opportunity costs







•	







•	Repeated interactions







•	







o	By increasing trust, repeated interactions between claimants also facilitates the development of institutions that can be used to monitor and enforce the terms of an agreement. 







o	























•	Findings







•	







•	Implications







•	







\section{Domestic Politics and Adherence to International Agreements}















Although leaders consider a host of domestic and international factors when negotiating international agreements, whether or not an agreement can be feasibly implemented depends on whether domestic audiences support the agreement. Although leaders are unlikely to sign agreements that they do not intend to implement due to the high domestic and international costs of reneging \citep[e.g., ][]{schelling1960, toft2003, zacher2001}, this does not ultimately guarantee that domestic audiences will end up supporting an agreement.















However, the fact that leaders expect an agreement to mitigate the domestic consequences of signing agreements does not g















•	Leaders do not have perfect information about the preferences of their constituents. Brexit example







•	







•	Constituents support an agreement does not mean they will support any agreement. 







•	







•	Probabilistic







•	







•	Shifts in preferences







•	







Leaders who sign agreements that are unpopular among their domestic supporters will be unable to obtain their support in the process of ratifying an agreement and adhering to it over the long term. 















% additional cites on cost of reneging: crescenzi2003, morrow2000, simmons2000















% Renegotiation – putnam1988, mattes2008, fearon1998, go back to methods chapter















% Compliance requires that states continue to view an agreements as preferable to nonagreement















% changes in preferences















% Changes in expected costs of conflict















% The more society benefits, the more segments of society benefit, the harder it is to break the bargaining range















% This is particularly true if the domestic regime supports the agreement















Ratification















With respect to ratification, the successful implementation of any agreement requires the cooperation of at least some domestic actors who have the power to stymie its entry into force. Following \citet{putnam1988}, I use the term ``ratification’’ to refer broadly to any process at the domestic level that is necessary to implement international agreements. This includes formal processes required for a treaty to enter into force, such as approval by a legislature, or informal processes by which other powerful veto players (e.g., the military, bureaucracies, or administration officials) must approve of an agreement in order for it to be implemented effectively. 















The implementation of an agreement is thus contingent on whether the leader can obtain the support of relevant domestic actors by convincing them that the outcome of an agreement is preferable to the status quo. The number, size, and constitution of these actors varies widely depending on domestic institutions and the content of an agreement. Some treaties may be subject to the direct approval of voters in the form of referenda or plebiscites and must therefore be broadly popular with the public as a whole in order to be implemented. In other cases, ratification depends on the support of elected officials, who in turn are beholden to the preferences of constituents and/or subject to lobbying by influential groups. In order to obtain domestic support for ratification, leaders must thus be able to piece together coalitions of influential actors that support the agreement \citep{fearon1998}.















Compliance















Although ratification of an agreement is necessary for its ultimate success, it by no means guarantees that it will prevail in the long term. One key factor in determining whether an agreement is self-enforcing is whether domestic audiences support its implementation over the long term \citep{fearon1998}. If the expected benefits of an agreement fail to materialize, domestic audiences may prefer that leaders attempt to renegotiate the terms of an agreement or else abrogate that agreement entirely \citep{fearon1998, putnam1988}. Moreover, if leaders do not cave to constituent pressure, they may be removed from office and replaced by a leader who is willing to adhere to those demands. In addition, if agreements are broadly unpopular with members of the selectorate, challengers may use this discontent to mobilize opposition to the current leadership and elicit a change in winning coalition \citep[e.g., ][]{colaresi2005, fearon1998, vasquez2009}.















% Redundant: One circumstance that can lead states to attempt to renegotiate agreements is a change in the preferences of domestic supporters. This may occur due to changes in the winning coalition or leaders in power. As noted in the previous chapter, if members of the winning coalition oppose an agreement, it may lead them to replace a leader with one who is willing to adhere to their preferences. Moreover, if an agreement is unpopular with influential elites or voters, this may provide opposition leaders with the chance to mobilize additional opposition and engineer a change in the winning coalition. If this occurs, the new winning coalition would be able to abrogate an agreement \citep[e.g., ][]{colaresi2005, vasquez2009}.















States may also de facto renege on an agreement due to principal-agent problems. The implementation of an agreement is often dependent on actors such as bureaucracies or the military. Even if an agreement enters into force legally, these actors may have the power to hinder their effective implementation \citep{putnam1988}. The implementation of policies may also be reliant on the explicit or implicit of cooperation of interest groups (exg., traders) that are influenced by the policy.















\section{Determinants of Domestic Support}







\begin{comment}







Whether domestic actors support an agreement is dependent on two factors: whether actors support the terms of the agreement and whether they trust their opponent to adhere to the agreement.

Incentives

First, domestic constituencies must believe that the benefits of implementing an agreement outweigh the costs. If leaders choose to pursue

•	Compositoin of coalition
•	
•	Preferences
•	
•	Domestic institutions
•	
Interest groups affected by the policy most are particularly important for ratification 

•	Separation of powers
•	
•	Lobbying
•	
•	Important actors
•	
•	Discipline w/in governing party
•	
•	Autonomy of central decision-makers
•	
Commitment Problems

Second, domestic constituencies must believe that their opponent will adhere to the terms of an agreement. 

Additional Considerations

•	Mobilization
•	
•	Feedback Loops
•	
o	Opportunity costs of claims lead to settlement
o	
o	Settlement leads to increased trade
o	
o	Increase in trade creates feedback effects
o	
	Development in Contested Regions
	
	Joint Development Projects
	
	Development of Institutions
	
o	Trade leads to treaties \citep{espey2004}
o	
o	Threatent – espey and towfique, stinnett2009, gartzke et al 2001
o	
•	Iteration
•	
First, leaders who settle claims with important economic partners should be less likely to be punished by domestic audiences. As noted above, economic

\section{Trade and Domestic Support for Ratification and Compliance}

Although the previous chapter demonstrated that interdependent states are more likely to sign agreements, it cannot be taken for granted that they will be ratified and implemented afterwards. 

[Agreement =/= Implementation]

•	Mobilization
•	
\end{comment}
Previous research has found that trade is associated with cooperation over river claims. States that trade more are more likely to sign treaties governing river claims \citep{dinar2011, espey2004, tir2009}.\footnote{\citet{dinar2011} find that very high levels of trade may decrease the probability of agreement by creating tensions between states \citep[e.g.][]{barbieri2002}. \citet{stinnett2009} find that, when states sign river treaties, highly interdependent states are more likely to establish formal international institutions as part of the agreement.}
	
	However, much of this research does not focus on whether states involved in river claims are more likely to sign agreements. Instead, it focuses purely on whether states sign treaties, which do not necessarily take place in the context of conflictual interactions. Even states that do not have grievances with each other have incentives to coordinate over the development and management of rivers in order to facilitate navigation, reduce pollution, and ensure that states receive enough water.  
	
	Don’t examine in the context of issue claims – mostly looking at treaty formation 
	
	Dinar and Dinar (2005) argue that a record of trade between states indicate a history of cooperation that may lower the threshold to enter into negotiations and eventually sign treaties.- brochmann and hensel
	
	Due to improvements in states’ abilities to communicate with one another and demonstrate resolve when they are economically interdependent, overcoming the sanctioning problem in multilateral cooperation should be easier.  zw2011
	
	Economics creates repeated interactions that make cooperation easier, brings greater benefits, and is less politically controversial – tir2009 
	
	Conflict de vries 1990, emphasize the particular elements of an inter-state trade regime thatbode well for a beneficial relationship among states (Gasiorowski 1986)
	
	Incentives
	
	Interdependence increases domestic support for the ratification and implementation of agreements both by increasing the expected utility of adhering to agreements and by resolving commitment problems that may prevent domestic actors from supporting an agreement. With respect to expected utility, trade provides domestic actors with incentives to support the implementation of an agreement by increasing the benefits of doing so. As argued in the previous chapter, economic actors who engage in high levels of trade stand to lose if military or diplomatic conflict emerges between the two disputants. Moreover, economic actors that forego trade with the opposing state due to the potential for losses stand to benefit from resolving the claim and thereby creating the potential to expand their operations. By increasing the incentives for these actors to support the ratification and implementation, interdependence expands the win-set or range of agreements that are acceptable domestic actors. Doing so increase the likelihood that the agreements leaders secure will satisfy the demands of these actors.
	
	
	
	Note that this strategy works not by changing the preferences of any domestic constituents, but rather by creating a policy option (such as faster export growth) that was previously beyond domestic Control…. For example, ‘in the Tokyo Round. nations used negotiation to achieve internal reform in situations where constituency pressures would otherwise prevent action without the pressure (and tradeoff benefits) that an external partner could provide.'58 Economic interdependence multiplies the opportunities for altering domestic coalitions (and thus policy outcomes) by expanding the set of feasible alternatives in this way-in effect, creating political entanglements across national boundaries. Thus, we should expect synergistic linkage(which is, by definition, explicable only in terms of two-level analysis) to become more frequent as interdependence grows. Putnam 447-448
	
	Commitment Problems
	
	Costs of defection
	
	Trade also helps alleviate commitment problems in four ways. First, because domestic actors stand to benefit more from an agreement with important economic partners, interdependence increases the costs of defecting from an agreement. In doing so, domestic audiences in one state have less reason to fear defection by the other state and therefore should assign a higher expected utility to adhering to the agreement.
	
	Repeated Interactions
	
	Second, trade facilitates repeated interactions between states which may then foster cooperation on other issues \citep{lerner1956, rosecrance1986, russet1963}. Repeated interactions between help establish trust between states by creating channels of communication, creating more predictable relationships, and reducing the costs of cooperation \citep{axelrod1984, blum2007, gartzke2001, keohane2001, kydd2001, moravcsik1997, russett2001}. In doing so, cooperation on some issues (in this case, trade) can lay the groundwork for cooperation over other issues. As \citet{blum2007} notes, managing issues that are ``not explicitly tied to resolution may also allow the parties to build trust, understanding, and consideration, without fearing that this more benign approach necessarily be perceived by their rival (or by hard-line domestic constituencies) as a willingness to make more concessions on the core contested issues’’ (p. 249).
	
	% Building on the positive empirical relationship between trade interdependence and interstate cooperation, some have argued that greater economic exchanges can contribute to increased interactions between states (McMillan 1997). These increased interactions can lower the transaction costs of reaching agreements in other issue areas by facilitating the use of issue linkages and side-payments (Gartzke, Li, and Boehmer 2001). Economic interdependence can also build an environment of trustworthiness, which is expected to encourage states to foster cooperative relations over the management of other issue areas, such as international rivers. In fact, empirical analysis shows that economically interdependent states are more likely to reach agreements over their international rivers (Espey and Towfique 2004; Tir and Ackerman 2009). 
	
	Contact Theory
	
	Third, bilateral trade leads to increased interactions between individuals from each country. Exposure to other groups facilitates the spread of ideas across groups and leads to the formation of personal connections with members of the other country. Increased mobility can also lead to migration between countries, creating familial ties within each and bringing foreigners into contact even with individuals who do not engage in trade with the other state. These positive contacts between individuals can help reduce intergroup hostility and prejudice and promote social trust \citep{pettigrew2008}. In doing so, these interactions can allay fears of defection by the other side and prevent or undo the formation of hostile images of the enemy that contribute to commitment problems \citep{vasquez2009}. % TODO – Vasquez cite
	
	Institutions
	
	Fourth, building trust between the disputants increases the prospects that agreements include stronger monitoring, enforcement, and conflict resolution mechanisms \citep{elhance2000, espey2004, stinnett2009, tir2011, verghese1993}. Because these institutions require delegating some degree of sovereignty over a state’s actions with respect to an issue claim, states will generally be wary of agreeing to enforcement actions. As \citet{stinnett2009} note, ``when states trust each other they will be more willing to accept the sovereignty costs that come from delegating authority to international institutions, ‘’ (p. 240). These institutions reduce uncertainty and increase the costs of defection, helping eliminate commitment problems. 
	
	Institutions increase incentives for cooperation
	
	Hence, river treaties can create positive-sum situations in which the incentives for cooperation become palpable (Benvenisti, 1996; Waterbury, 2002)  even in the face of problematic relationships (e.g. the recent Indus River treaty).
	
	Mansifeild and pollins 2003 heightened trade facilitates environmental treaty formation and acts as a contract enforcing mechanism (Neumayer 2002b; Stein 2003).New Perspectives on Economic Exchange and Armed Conflict.
	
	Monitoring institutions can also improve compliance with river treaties. Formal mechanisms for monitoring the river can provide early warning of violations, which will prompt quicker enforcement actions (Benvenisti 1996). Even in cases where monitoring institutions do not track the actions of individual states, the information they provide can raise the visibility of environmental problems and draw the attention of domestic actors to the consequences of non-compliance (Dai 2005). This information will provide leverage to domestic interest groups favoring cooperation when they pressure their governments to resume compliance. Public pressure is especially important for implementing environmental agreements (Haas, Keohane, and Levy 1993).
	
	% Even in cases where monitoring institutions do not track the actions of individual states, the information they provide can raise the visibility of environmental problems and draw the attention of domestic actors to the consequences of non-compliance (Dai 2005). This information will provide leverage to domestic interest groups favoring cooperation when they pressure their governments to resume compliance. Public pressure is especially important for implementing environmental agreements (Haas, Keohane, and Levy 1993). % Additional citations: dinar 2011, dai2005, Benvenisti, 1996; Waterbury, 2002 – from one of the river articles
	

\begin{comment}
%%%%% Interative agreements
%%%%% Feedback effects



- Introduction
	- Introductory Paragraph
	- Theory
	- Findings
	- Implications
Domestic Politics and Compliance with International Agreements %*** Putnam 1988 /Fearon 1998
	- Domestic Support K2 Adherence
		- Domestic Politics and Ratification
		- Domestic Politics and Compliance
		- Domestic Politics and Institutions/Enforcement
	- Determinants of Domestic Support
		- Satisfied with terms of the agreement
		- Commitment Problems
Trade And Domestic Support for Adherence
	- Trade Increases Incentives to Support Agreement
		- Opp Costs Increase Incentives to Comply
	- Trade Resolves Commitment Problems
		- Trust Mechanism
			- Trade Increases Trust
				- Repeated Interactions
				- Higher costs for other side
				- Communication/Information
- Hypotheses
	- Leader Removal
	- Coalition Change
	- Ratification
	- Compliance
	- Claim Duration
	- Signing to MIDs
- Research Design
- Analysis
- Conclusion
	- Disaggregate Mechanisms
	- Issue Discrepancies
	
	
	
\section{Analysis – Ratification and Compliance} %----------------------------------------------------------------------
	Dependent Variables
		•	Ratification
		•	Compliance
		•	Claim Duration
	Independent Variables
	
	Results
	
\section{Analysis – Claim Duration}
	\end{comment}